
\documentclass[letterpaper,11pt]{article}
\newlength{\outerbordwidth}
\pagestyle{empty}
\raggedbottom
\raggedright
\usepackage[svgnames]{xcolor}
\usepackage{framed}
\usepackage{tocloft}
\usepackage[backend=bibtex,maxnames=99]{biblatex}
\usepackage{hyperref}

\usepackage{graphicx}


%-----------------------------------------------------------
% Hyperlink setup
\hypersetup{
    colorlinks=true,        % false: boxed links; true: colored links
    linkcolor=red,          % color of internal links
    citecolor=green,        % color of links to bibliography
    filecolor=magenta,      % color of file links
    urlcolor=red            % color of external links
}

%-----------------------------------------------------------
%Edit these values as you see fit

% Width of border outside of title bars
\setlength{\outerbordwidth}{3pt}
% Outer background color of title bars (0 = black, 1 = white)
\definecolor{shadecolor}{gray}{0.75}
% Inner background color of title bars
\definecolor{shadecolorB}{gray}{0.93}


%-----------------------------------------------------------


\title{ {{course_id}} Grades }
\author{ {{student_id}} }
\date{\today}

\begin{document}
\maketitle

\begin{tabular}[h]{|l||c|c|c|c|}
\hline
\textbf{Category} & \textbf{Score} & \textbf{Max} & \textbf{Weight} &
    \textbf{Weighted Frac} \\

\hline
\textbf{ {{row[0]}} }

    & {{x}}
 \\

\hline
\end{tabular}

\vspace{1em}
\textbf{Total weighted average, raw:} {{student_wavg*100}}\%

\textbf{Best weighted average, raw:} {{max_wavg*100}}\%

\textbf{Letter Grade, raw:} {{student_letter_grade_raw}}

\textbf{Curve, $100 - \mathrm{best}$:} {{curve*100}}\%

\textbf{Total weighted average, curved:} {{(student_wavg+curve)*100}}\%

\textbf{Letter Grade, curved:} {{student_letter_grade_curved}}


\begin{figure}[h]
\centering
\includegraphics[scale=0.5]{student-letter-grade-dist.eps}
\end{figure}




\vspace{1em}\hrulefill

\section*{ {{category.capitalize()}} }



\subsection*{ {{asgn_id}} }

\begin{tabular}[h]{|l||c|c|c|c|c|c|c|}
\hline
\textbf{Question} & \textbf{Score} & \textbf{Points} & \textbf{Mean} & \textbf{STD}
                  & \textbf{Max} & \textbf{$>60\%$} & \textbf{$>60\%$} \\

\hline
\textbf{ {{asgn['questions'][i]}} } &
    
    {\color{red} nil} &
    
    {{sg['scores'][i] | round(3)}} &
    
    {{asgn['points'][i] | round(3)}} &
    {{stats[asgn_id][0][i] | round(3)}} &
    {{stats[asgn_id][1][i] | round(3)}} &
    {{stats[asgn_id][2][i] | round(3)}} &
    {{stats[asgn_id][3][i] | round(3)}} &
    {{stats[asgn_id][4][i] | round(3)}} \\

\hline
\textbf{Total} &
    
    {\color{red} nil} &
    
    {{sum(sg['scores']) | round(3)}} &
    
    {{sum(asgn['points']) | round(3)}} &
    {{stats[asgn_id][5] | round(3)}} &
    {{stats[asgn_id][6] | round(3)}} &
    {{stats[asgn_id][7] | round(3)}} &
    {{stats[asgn_id][8] | round(3)}} &
    {{stats[asgn_id][9] | round(3)}} \\
\hline
\end{tabular}





\end{document}
