\documentclass[prl,aps,tighten,amsmath,amssymb,floatfix]{revtex4-1}

\usepackage{graphicx}

\newcommand{\midrule}{\hline}
\newcommand{\bottomrule}{\hline\hline}

\begin{document}
\title{The experiment plan for the beamtime 2020-1-XPD}
\date{\today}
\maketitle

\section{Summary}
The beamtime starts at 8:00 am on Feb 14, 2020 and ends at 8:00 am on Feb 17, 2020.
There are 2 samples in total.
The estimated experiment time is 3.2 hour.
The TABLE.~\ref{tab:exps} shows the summary of the experiment plans for this beamtime.

\begin{table}[htpb]
\caption{Summary of the experiment plans.}
\label{tab:exps}
\resizebox{\textwidth}{!}{
\begin{tabular}{llllll}
\toprule
serial id & project leader & number of samples & measurement &     devices & estimated time (h) \\
\midrule
        1 &          kseth &                 2 &       Tramp &  cryostream &                3.2 \\
\bottomrule
\end{tabular}
}
\end{table}

\subsection{Tasks before the beamtime}
\begin{enumerate}
\item (Exp. 1) todo something
\end{enumerate}
\section{Details}

\subsection{Experiment 1}
\subsubsection{Objective}
temperature ramping PDF of one WO3 film (100, 300K, 10K)
\subsubsection{Samples}
WO3 film, glass subtrate
\subsubsection{Preparation}
\begin{enumerate}
\item films will be made by kriti
\end{enumerate}
\subsubsection{Shipment}
\begin{enumerate}
\item seal and ship to CU
\item carry to the beamline
\end{enumerate}
\subsubsection{Measurement}
\begin{enumerate}
\item load samples on the holder
\item scan the holder to locate the samples
\item take room temperature measurement of sample and the subtrate
\item ramp down temperature to 100K
\item ramp up, measure PDF at temperature 100K ~ 300K, 10K stepsize, 1 min exposure
\end{enumerate}

\end{document}